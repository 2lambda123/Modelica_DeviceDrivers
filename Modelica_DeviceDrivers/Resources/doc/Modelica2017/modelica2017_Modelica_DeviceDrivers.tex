% paper proposal for the Modelica 2017 Conference

%%% By default the modelica LaTeX class uses bibtex and natbib for refrences
\documentclass{resources/modelica}
%%% As alternative also for unicode and @online support
%%% use the more modern biber and biblatex instead
%\documentclass[backend=biber]{modelica}
%\addbibresource{example-paper.bib}
%\usepackage[utf8]{luainputenc} % utf8 input encoding that should work for both pdflatex and lualatex, but might not be available in every installation
\usepackage[utf8]{inputenc} % utf8 input encoding which should work with pdflatex, but not lualatex

\hypersetup{%
	pdftitle  = {Towards a Standard-Conform, Platform-Generic and Feature-Rich Modelica Device Drivers Library},
	pdfauthor = {Bernhard Thiele, Thomas Beutlich, Volker Waurich, Martin Sjölund,
	Tobias Bellmann}, pdfsubject = {12th International Modelica Conference 2017},
  pdfkeywords = {Modelica, embedded systems, real-time simulation},
	colorlinks,
	linkcolor=black,
	urlcolor=black,
	citecolor=black,
	pdfpagelayout = SinglePage,
	pdfcreator = pdflatex,
	pdfproducer = pdflatex}


\newcommand{\BTHI}[1]{{\color{blue}{$\parallel_\textrm{BTHI}$#1$\parallel$}}}
\newcommand{\TBEU}[1]{{\color{orange}{$\parallel_\textrm{TBEU}$#1$\parallel$}}}
\newcommand{\VWAU}[1]{{\color{red}{$\parallel_\textrm{VWAU}$#1$\parallel$}}}
\newcommand{\MSJO}[1]{{\color{green}{$\parallel_\textrm{MSJO}$#1$\parallel$}}}
\newcommand{\TBEL}[1]{{\color{magenta}{$\parallel_\textrm{TBEL}$#1$\parallel$}}}



% begin the document
\begin{document}
\thispagestyle{empty}

\title{Towards a Standard-Conform, Platform-Generic and Feature-Rich Modelica Device Drivers Library}
\author[1]{Bernhard Thiele}
\author[2]{Thomas Beutlich}
\author[3]{Volker Waurich}
\author[1]{Martin Sjölund}
\author[4]{Tobias Bellmann}
\affil[1]{PELAB, Linköping University, Sweden,
{\small\texttt{\{bernhard.thiele,martin.sjoelund\}@liu.se}}}
\affil[2]{ESI ITI GmbH, Germany, {\small\texttt{thomas.beutlich@esi-group.com}}}
\affil[4]{Institute of System Dynamics and Control, DLR, Germany, {\small\texttt{tobias.bellmann@dlr.se}}}


\date{} % <--- leave date empty
\maketitle\thispagestyle{empty} %% <-- you need this for the first page
\abstract{%
Tbd.
}

\noindent\emph{Keywords: keyword1, keyword2}

\section{Introduction}
\BTHI{TODO: Bernhard, Tobias}\\

History:
\begin{itemize}
  \item Interactive Simulations and advanced Visualization with Modelica and
  Modelica \cite{Bellmann2009}
  \item Modelica for Embedded Systems \cite{Elmqvist2009}
\end{itemize}

\section{Modelica\_DeviceDrivers}
\BTHI{TODO: Bernhard, Thomas, Volker}
\subsection{Enhanced Tool Support}
\subsection{Cross-Platform Support}
\subsection{Library Structure}
\subsection{Interfaces}

\subsection{Features}

Features (aide-mémoire):
\begin{itemize}
  \item SerialPackager
  \item Communication
    \begin{itemize}
      \item UDP
      \item Shared Memory
      \item Softing CAN and Socket CAN
      \item RS232
    \end{itemize}
  \item input devices
    \begin{itemize}
      \item keyboard
      \item joystick/gamepad
      \item 3Dconnexion Spacemouse.
    \end{itemize}
  \item Operating System
    \begin{itemize}
      \item Realtime Synchronization
      \item Random numbers
    \end{itemize}
  \item Hardware I/O
    \begin{itemize}
      \item Comedi
    \end{itemize}
\end{itemize}


\noindent  New features:
\begin{itemize}
  \item Big endian
  \item Serial port on Win
  \item TCP/IP client
  \item LCM / UDP Broadcast
  \item Bluetooth
  \item MQTT
  \item \ldots
\end{itemize}

\section{Modelica Standard Compliance}
\BTHI{TODO: Thomas, Bernhard}

\subsection{Enhancements}

\subsection{Pitfalls and Open Issues}

Aide-mémoire:
\begin{itemize}
  \item Serialpackager
  \item Automatic buffer size
  \item External objects in equation
  \item Construction of external objects in record
  \item Linking to platform-dependent system libraries
  \item Missing fixed attribute for String
  \item Support of several include directories
\end{itemize}

\section{Applications}

\subsection{Arduino}
\BTHI{TODO: Volker}

The arduino is an open-source electronics platform which makes it very easy to read sensors, process the data and send it to another device via a serial connection \citep{arduino:2016}.
With the help of Modelica Device Drivers serial port implementation, the arduino can be utilized to make sensor data available in a real-time Modelica model.
Using different kinds of potentiometers, it is possible to build customized control devices.
As an exemplary application, pedals for a driving simulator can be equipped with a rotary potentiometer in order to measure the displacement.
This data can be send via a serial connection to a \textit{Blocks.Comunication.SerialPortReceive} in order to drive a virtual vehicle.
Hence, expensive or unavailable input devices can be substituted by self-built constructions.
By using a bluetooth module, a wireless connection between arduino and the simulator is set up easily.
\VWAU{I could add arduino code or an arduino circuit diagram here.}

\subsection{Arduino, Raspberry PI, embedded control}
\BTHI{TODO: Bernhard, Martin}

\subsection{HID Joystick}
\BTHI{TODO: Volker }

\subsection{DLR Demonstrators}
\BTHI{TODO: Tobias}

\section{Outlook}
\BTHI{TODO: Bernhard, Thomas, Volker}

\section*{Acknowledgements}


%%% choose one of the following: %%%
%% References using bibtex (default)
\bibliography{modelica2017_Modelica_DeviceDrivers}

%% References using biber and biblatex
%\printbibliography

\end{document}
